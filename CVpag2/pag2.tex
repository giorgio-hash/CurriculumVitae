\documentclass[a4paper]{letter}
\usepackage{wallpaper}
\usepackage{geometry}
\usepackage{xcolor}
\usepackage[T1]{fontenc}
\usepackage[scaled]{helvet}
\usepackage{fontawesome5}
\usepackage{hyperref}
\usepackage[none]{hyphenat}
\usepackage{tikz}

\renewcommand{\familydefault}{\sfdefault}


\geometry{
  a4paper, % Set paper size to A4
  %total={210mm,297mm}, % A4 paper dimensions
  left=20pt,
  right=20pt,
  top=0pt,
  bottom=0pt,
  nohead,
  nofoot,
  nomarginpar
}

\ThisCenterWallPaper{1.02}{../risorse/cvbg.jpg}

\begin{document}

\begin{minipage}[t]{0.30\textwidth}
\setlength{\baselineskip}{1.5\baselineskip}
\color{white}


\end{minipage}
\hfill
\begin{minipage}[t]{0.65\textwidth}
\setlength{\baselineskip}{1.4\baselineskip}

\vspace{2 mm}
{\large Progetti Triennale}
\rule{\linewidth}{0.4pt}

Focus sull'esplorazione dei linguaggi ed applicazione dei protocolli di rete.

\vspace{2 mm}
{\large \textbf{Proj\_GOLDUCK}}

{\small Web-app per monitoraggio real-time di una gara di Orienteering, hosting su AWS. }

\begin{itemize}
    \item Back-end in Node.js attraverso moduli low-code Lambda di Amazon
    \item Storage XML ed utilizzo di JSON per informazioni real-time relative all'evento
    \item Front-end in Flutter con ricezione dati periodica in logica pull
    \item Deployment su Amazon Web Service
\end{itemize}
\vspace{1 mm}

{\large \textbf{Docker Escape}}

{\small Tesi triennale dove vengono descritte vulnerabilità note di Docker e possibili mitigazioni. }

\begin{itemize}
    \item Creazione di un laboratorio di pentesting con macchine virtuali in rete dedicata
    \item Approfondimento delle API di Docker ed interazione tramite curl e script
    \item Approfondimento dei sistemi Linux e dei moduli Seccomp, SELinux, AppArmor
    \item Produzione di script in Python3 (CPython) e C
\end{itemize}
\vspace{1 mm}

\vspace{1 mm}
{\large Progetti Magistrale}
\rule{\linewidth}{0.4pt}

Focus sul processo di realizzazione, testing e deployment di applicativi.

\vspace{2 mm}
{\large \textbf{ServeEasy}}

{\small Sistema a microservizi per la gestione di una attività di ristorazione. }

\vspace{1 mm}
\begin{itemize}
    \item Back-end e gateway in Spring Boot con integrazione dipendenze Maven
    \item Deployment della rete in Docker Compose con monitoraggio "healthcheck"
    \item Storage con MariaDB, gestione dati JSON tramite REST e Kafka Zookeeper, monitoraggio del flusso dati con Kafdrop e PHPMyAdmin
    \item Repository multi-repo con CI/CD Github Actions per testing d'integrazione e container shipping dei microservizi verso DockerHub
    \item Controllo qualità con PMD, SpotBugs, CheckStyle e sicurezza con SonarQube
\end{itemize}

\vspace{1 mm}
{\large \textbf{Programmazione Genetica per search-based falsification testing}}

{\small Applicazione di AI su modello cyber-fisico per falsificare le specifiche di un sistema automotive. }

\vspace{1 mm}
\begin{itemize}
    \item Esplorazione del paradigma Genetic Programming ed analogie evolutive
    \item Applicazione del framework "Hecate S-Taliro" e moduli MATLAB gaoptimset
\end{itemize}

\vspace{1 mm}
{\large \textbf{Beemediate}}

{\small Middleware di integrazione per mediare tra gestionali Odoo e SAP.}

\vspace{1 mm}
\begin{itemize}
    \item Caso studio con protocollo FTP tra cliente e fornitore
    \item Architettura esagonale, mapping tra XML-RPC Odoo ed XML-OpenTrans
    \item DevSecOps con SonarQube Cloud Quality Check
    \item Verifica formale della business logic tramite OpenJML 
\end{itemize}

\end{minipage}

\end{document}
