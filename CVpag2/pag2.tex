\documentclass[a4paper]{letter}
\usepackage{wallpaper}
\usepackage{geometry}
\usepackage{xcolor}
\usepackage[T1]{fontenc}
\usepackage[scaled]{helvet}
\usepackage{fontawesome5}
\usepackage{hyperref}
\usepackage[none]{hyphenat}
\usepackage{tikz}

\renewcommand{\familydefault}{\sfdefault}


\geometry{
  a4paper, % Set paper size to A4
  %total={210mm,297mm}, % A4 paper dimensions
  left=20pt,
  right=20pt,
  top=0pt,
  bottom=0pt,
  nohead,
  nofoot,
  nomarginpar
}

\ThisCenterWallPaper{1.02}{../risorse/cvbg.jpg}

\begin{document}

\begin{minipage}[t]{0.30\textwidth}
\setlength{\baselineskip}{1.5\baselineskip}
\color{white}


\end{minipage}
\hfill
\begin{minipage}[t]{0.65\textwidth}
\setlength{\baselineskip}{1.4\baselineskip}

\vspace{2 mm}

\makeatletter
\newcommand{\ProgettiTriennaleBox}{%
    \begin{tikzpicture}[remember picture,overlay]
        \coordinate (boxtop) at ($(current page.north west)+(20pt+0.5cm,-\dimexpr\pagetotal+1.5\baselineskip\relax)$);
        \node[anchor=north west, draw={rgb,255:red,18;green,44;blue,71}, fill={rgb,255:red,18;green,44;blue,71}, text=white, rounded corners, inner sep=6pt, minimum width=\dimexpr\paperwidth-40pt-0.5cm\relax] at (boxtop) {%
            \begin{minipage}[t]{\dimexpr\paperwidth-40pt-0.5cm-12pt\relax}%
                {\large \color{white}Progetti Triennale}\color{white}\rule{\linewidth}{0.4pt}
            \end{minipage}%
        };
    \end{tikzpicture}%
}
\makeatother

\ProgettiTriennaleBox

\vspace{0.8 cm}

Focus sull'esplorazione dei linguaggi ed applicazione dei protocolli di rete. Durante la tesi ho avuto il mio primo contatto col mondo della cybersecurity.

\vspace{2 mm}
{\large \textbf{Proj\_GOLDUCK}}

{\small Web-app per monitoraggio real-time di una gara di Orienteering. }

\begin{itemize}
    \item Hosting della Back-end su Amazon Web Service, moduli Lambda low-code.
    \item Storage XML ed utilizzo di JSON per informazioni real-time relative all'evento.
    \item Front-end in Flutter con ricezione dati in logica push.
\end{itemize}
\vspace{1 mm}

{\large \textbf{Docker Escape}}

{\small Tesi triennale dove vengono descritte vulnerabilità note di Docker e possibili mitigazioni. }

\begin{itemize}
    \item Creazione di un laboratorio di pentesting con macchine virtuali in rete dedicata.
    \item Approfondimento delle API di Docker ed interazione tramite curl e script.
    \item Approfondimento dei sistemi Linux e dei moduli Seccomp, SELinux, AppArmor.
    \item Produzione di script in Python3 (CPython) e C.
\end{itemize}
\vspace{1 mm}

\vspace{0.15 cm}

\makeatletter
\newcommand{\ProgettiMagistraleBox}{%
    \begin{tikzpicture}[remember picture,overlay]
        \coordinate (boxtop) at ($(current page.north west)+(20pt+0.5cm,-\dimexpr\pagetotal+1.5\baselineskip\relax)$);
        \node[anchor=north west, draw={rgb,255:red,18;green,44;blue,71}, fill={rgb,255:red,18;green,44;blue,71}, text=white, rounded corners, inner sep=6pt, minimum width=\dimexpr\paperwidth-40pt-0.5cm\relax] at (boxtop) {%
            \begin{minipage}[t]{\dimexpr\paperwidth-40pt-0.5cm-12pt\relax}%
                {\large \color{white}Progetti Magistrale}\color{white}\rule{\linewidth}{0.4pt}
            \end{minipage}%
        };
    \end{tikzpicture}%
}
\makeatother

\ProgettiMagistraleBox

\vspace{0.8 cm}

Focus sul processo di realizzazione, testing e deployment di applicativi, cercando di applicare nozioni di InfoSec e Security By Design.

\vspace{2 mm}
{\large \textbf{ServeEasy}}

{\small Sistema a microservizi per la gestione di una attività di ristorazione. }
\vspace{1 mm}
\begin{itemize}
    \item Architettura a microservizi: back-end e gateway in Spring Boot (Java).
    \item Deploy della rete con Docker Compose; monitoraggio dello stato dei servizi.
    \item Storage su MariaDB, comunicazione sulla rete con paradigma pub-sub e REST.
    \item Pipeline CI/CD con GitHub Actions per test d'integrazione e rilascio dei container.
    \item Controllo qualità e sicurezza del codice mediante strumenti automatici di analisi.
\end{itemize}

\vspace{1 mm}
{\large \textbf{Beemediate}}

{\small Caso studio con azienda. Middleware di integrazione per ordini tra cliente Odoo e fornitore SAP.}

\vspace{1 mm}
\begin{itemize}
    \item Progetto individuale.
    \item Contatto col cliente per raccolta, elaborazione e verifica delle specifiche.
    \item Verifica formale della business logic per garantire il rispetto delle specifiche.
    \item Pipeline CI/CD con Github Actions per gestione delle vulnerabilità, requisiti di qualità, analisi del codice e delle dipendenze.
\end{itemize}

\vspace{1 mm}
{\large \textbf{Programmazione Genetica per search-based falsification testing}}

{\small Applicazione di AI su modello cyber-fisico per falsificare le specifiche di un sistema automotive. }

\vspace{1 mm}
\begin{itemize}
    \item Esplorazione del paradigma Genetic Programming ed analogie evolutive
    \item Applicazione del framework "Hecate S-Taliro" e moduli MATLAB gaoptimset
\end{itemize}

\end{minipage}

\end{document}
