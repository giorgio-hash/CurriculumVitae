\documentclass[a4paper]{letter}
\usepackage{wallpaper}
\usepackage{geometry}
\usepackage{xcolor}
\usepackage[T1]{fontenc}
\usepackage[scaled]{helvet}
\usepackage{fontawesome5}
\usepackage{hyperref}
\usepackage[none]{hyphenat}
\usepackage{tikz}

\renewcommand{\familydefault}{\sfdefault}

\geometry{
  a4paper, % Set paper size to A4
  %total={210mm,297mm}, % A4 paper dimensions
  left=20pt,
  right=20pt,
  top=0pt,
  bottom=0pt,
  nohead,
  nofoot,
  nomarginpar
}

\ThisCenterWallPaper{1.02}{../risorse/cvbg.jpg}

\begin{document}

\begin{minipage}[t]{0.30\textwidth}
\setlength{\baselineskip}{1.5\baselineskip}
\color{white}


\begin{figure}
    \quad \quad \begin{tikzpicture}
        \clip (2,1)  circle (2.2cm) ;
        \node[anchor=center] at (2,1) {\includegraphics[scale=.4]{../risorse/foto.png}}
    \end{tikzpicture}
\end{figure}


\vspace{1.5mm}
{\large Dati personali e recapiti}

\vspace{2.2mm}
\faBaby \quad Nato il 11/09/2000 a Segrate (MI)

\vspace{2.2mm}
\begin{tabular}{@{}l@{\quad}p{0.8\textwidth}}
   \faMapMarker & Residente a \\
                & Bellinzago Lombardo (MI), \\
                & via Lombardia 19, 20060 \\
\end{tabular}
\vspace{1.5mm}

\faPhone \quad +39 3664181411

\faEnvelope \quad \href{mailto://gchirico28@gmail.com}{gchirico28@gmail.com}

\rule{\linewidth}{0.4pt}

{\large Link}

\faLinkedin \quad \href{https://tinyurl.com/gchirlinkedin}{https://tinyurl.com/gchirlinkedin}

\faGithub \quad \href{https://github.com/giorgio-hash}{github.com/giorgio-hash}

\rule{\linewidth}{0.4pt}

{\large Tecnologie front-end}

\faCode \quad HTML, CSS (ottimo)

\faCode \quad Bootstrap (base)

\faCode \quad JavaScript (ottimo)

\faCode \quad Java JFC, Swing (intermedio)

\faCode \quad Flutter (intermedio)



\rule{\linewidth}{0.4pt}

{\large Tecnologie back-end}

\faCode \quad Docker (ottimo)

\faCode \quad Docker Compose (ottimo)

\faCode \quad Java Spring Boot (ottimo)

\faCode \quad Node.js (intermedio)

\faCode \quad PHP (intermedio)

\faCode \quad Python Flask (base)

\rule{\linewidth}{0.4pt}


{\large Tecnologie Query}

\faCode \quad SQL (ottimo)

\faCode \quad Transact-SQL (base)

\faCode \quad XML (intermedio)

\faCode \quad XPath (intermedio)

\faCode \quad Python Pandas (base)

\rule{\linewidth}{0.4pt}

{\large Project management}

\faCode \quad Git Bash (ottimo)

\faCode \quad GitHub (intermedio)

\faCode \quad DockerHub (intermedio)

\faCode \quad UML (intermedio)



\end{minipage}
\hfill
\begin{minipage}[t]{0.65\textwidth}
\setlength{\baselineskip}{1.4\baselineskip}

{\tiny Autorizzo il trattamento dei dati personali presenti nel CV ai sensi del D.lgs.2018/101 e del GDPR (Regolamento UE 2016/2019)}
\vspace{0.3cm}


{\huge Giorgio Chirico}

{\large Dottore in Ingegneria Informatica (classe L-08)}

\vspace{0.5cm}
 
Formazione da ingegnere informatico, volta alla realizzazione di soluzioni ottimizzate, e passione per la cybersecurity. Orientato ad un aggiornamento costante sulle tecnologie attuali e ad un impiego sicuro di queste. Interessato a lavorare nello sviluppo di applicativi e nella messa in sicurezza dei sistemi.

\vspace{0.5cm}

{\large Formazione}
\rule{\linewidth}{0.4pt}

{\large \textbf{Laurea Magistrale in Ingegneria Informatica}}

{\small Università di Bergamo, Dalmine (BG)}
\begin{itemize}
    \item Settembre 2023 - oggi
    \item Frequentante full-time. Specializzazione in Sistemi In Rete
    \item Formazione IT: architetture per applicativi in rete e calcolo distribuito, AI e machine learning, programmazione ottimizzata (algoritmi, pattern, debug, analisi memoria x86), metodi formali e testing, progettazione compilatori
    \item Formazione ingegneristica: basi di crittografia, basi di sicurezza e protezione dati, ontologie, telecomunicazoni, gestione sistemi ICT, teoria dell'informazione e della trasmissione codificata, modelli predittivi e di ottimizzazione matematica
\end{itemize}

{\large \textbf{Laurea Triennale in Ingegneria Informatica} \small (85/110)}

{\small Università di Bergamo, Dalmine (BG)}
\begin{itemize}
    \item Settembre 2019 - Settembre 2023.
    \item Tesi: \href{https://github.com/giorgio-hash/tesi-triennio}{"Vulnerabilità di Fuga Da Docker"}
    \item Formazione IT: Web-app su cloud AWS, sistemi distribuiti, basi di dati non relazionali, database trigger, Android, programmazione multi-thread, ingegneria del software (modelli Waterfall, Agile, Scrum), Assembly su MIPS
    \item Formazione ingegneristica: statistica e modelli stocastici, elettrotecnica, elettronica analogica e digitale, controllo automatico, Risk Management
\end{itemize}

{\large \textbf{Diploma Tecnico Perito Informatico} \small (79/100)}

{\small Istituto Guglielmo Marconi, Gorgonzola (MI)}
\begin{itemize}
    \item Settembre 2014 - Giugno 2019
    \item basi di dati relazionali, linguaggi procedurali, programmazione web statica e dinamica, progettazione di sistemi sicuri per servire contenuti, XAMPP
\end{itemize}

\vspace{0.5cm}

{\large Esperienza Lavorativa}
\rule{\linewidth}{0.4pt}

{\large \textbf{Programmatore Java, Scuola Lavoro}}

{\small  Azienda IT consulting \textit{Open Gate}, Gorgonzola (MI)}

\begin{itemize}
    \item Giugno 2017 - Luglio 2017
    \item Lavoro di gruppo per produrre un videogioco 2D in Java, dotato di GUI
\end{itemize}

{\large \textbf{Lezioni private, autonomo}}

{\small In remoto e a domicilio}

\begin{itemize}
    \item 2020 - oggi
    \item Aiuto ai ragazzi delle superiori (e dell'università) nello studio e nel loro orientamento. Enfasi sull'empatia, mantenendo il ruolo professionale di guida.
\end{itemize}

\end{minipage}

\end{document}
